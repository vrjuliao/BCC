A etapa experimental demandou a criação de \textit{scripts} capazes de criarem instâncias do problema de forma aleatória. Tais instâncias foram executadas quinze vezes, sendo que gerou-se dez instâncias para cada uma das seguintes tamanhos de Sukodu($n$): 4, 6, 8, 9. Após tal etapa, pôde-se criar uma tabela para observar o comportamento das médias e do desvio padrão de cada uma das instâncias, permitindo a validação do teste. Além disso pode-se observar a taxa de acertos geral para as instâncias aleatórias geradas, assim, observando uma taxa de acertos de 100\% através da execução do comando de execução \textit{./tp3 entrada.txt -t} que exibe se a solução foi encontrada ou não (s/n)e o tempo de execução em nanosegundos.

Para os dois primeiros gráficos mostrados na Figura \ref{fig:grafico1}, é possível notar o comportamento não linear tanto para o gráfico de comportamento médio (esquerda) quando para o gráfico de desvio padrão (direita). Tal fato é um bom argumento justificativo a cerca do provável comportamento exponencial da solução. Em que é possível ainda, inferir uma provável "assíntota" que representa o rápido crescimento dos valores da função para as entradas de tamanho $n$.


Ainda é possível observar que o gráfico de desvio padrão também tem o mesmo comportamento, entretanto para valores muito baixos numericamente, se comparados aos valores médios. Uma vez que os casos de teste foram executados muitas vezes, a variância dos resultados é diminuída, fato explicado pela lei dos grades números. Isso é suficiente para demonstrar que há um comportamento possivelmente exponencial das solução apresentada.