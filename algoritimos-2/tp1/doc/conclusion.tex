Conclui-se portanto, que a escolha do algoritmo de \textit{backtrackin} é boa para entradas pequenas, visto a baixa quantidade de saídas não solucionadas em um tempo aceitável. Entretanto, pela análise dos gráficos, pode observar que essa não é uma boa metodologia de solução caso deseja-se executar uma instancia de tamanho muito grande, dado o comportamento exponencial da solução.

Mostrou-se ainda que é possível resolver um problema do mundo real, como um jogo de Sudoku, através da redução a soluções anteriormente conhecidas outros tipos de casos. É necessário ressaltar que a solução encontrada nesse trabalho pertence à classe de problemas NP-Completo, uma vez que coloração de grafos está nessa mesma classe e foi utilizado como forma de solução.