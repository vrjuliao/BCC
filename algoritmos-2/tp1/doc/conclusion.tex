Através da análise assintótica das duas funções e da observação das respostas
de tempo de execução da solução para a entrada passada, pôde-se concluir
que há uma grande diferença entre os tempos de obtenção da maior string e 
do processamento de sufixos.
Entretanto pode-se realizar um experimento com diversas entradas (de tamanhos
variados) para observar os tempos de execução de cada uma das funções e então
traçar um gráfico.
Nesse experimento, o comportamento do gráfico que computa os sufixos, terá uma
curva que remete à uma função polinômial quadrática.
Já a obtenção da maior substring que se repete o comportamento do gráfico é um
polinômio de grau 1.

A elaboração desse trabalho mostrou quão rápido é a manipulação de textos utilizando
uma árvore de sufixos compacta.
Apesar do custo elevado de construção da árvore apresentado nesse trabalho, existem
alternativas de construção da árvore com um custo menor que $O(n^2)$.
Esse fato torna a árvore de sufixos extremamente útil e aplicável em manipulações de
textos.