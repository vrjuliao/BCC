Como a solução desse problema consiste em duas etapas que são executada
separadamente, essa seção detalhará cada uma das etapas e suas respectivas
peculiaridades.

\subsection{Construção da árvore} 
Esse é o passo fundamental do problema, uma vez que para realizar qualquer
operação na árvore de sufixos, é necessário antes de tudo criar a
estrutura.
O algoritmo de montagem da árvore foi inspirado na explicação de um vídeo
\footnote{https://www.youtube.com/watch?v=VA9m\_l6LpwI}
e tem como ponto chave o algoritmo de busca de um a trie.
Dado um sufixo $s_i$ do texto, o algoritmo tenta realizar \textit{matchs}
parciais, entre algum prefixo de $s_i$ e o texto representado pelos nós
da árvore.
Ocorrendo um encontro parcial entre $s_i$ e um nó $N$, $s_i$ é atualizada, retirando
o prefixo que fora compatível com $N$, então reinicia-se a busca entre os
filhos de $N$ recursivamente.
Quando a busca se depara com uma substring $s_i$ (atualizada) que não
houve um \textit{match} parcial, então um novo nó é criado para tal substring.
Nesse cenário, o último nó o qual houve um encontro parcial de caracteres se
torna o nó "pai" do novo nó.
Em resumo o algoritmo de busca e construção consiste em consumir prefixos de
$s_i$, que é atualizado a cada etapa que o prefixos é encontrado no nó,
e então sufixo restante que não fora compatível é adicionado como um nó folha.
O nó inicial da busca é a raiz da árvore, e assume-se por nulidade que na raiz
já ocorre um match parcial.

Há um \textit{border case} na inserção que merece atenção:
Quando ocorre um \textit{match} parcial tanto para uma substring $ss$ de $s_i$
quanto para o nó $t$, ou seja, a substring $ss_t$ contida no nó não é uma
substring de $s_i$, então o $t$ deverá ser dividido em dois nós (um pai $t_p$ e
um filho $t_f$).
Nesse caso, a divisão será de tal forma que:
\begin{itemize}
    \item $t_p$ conterá o match parcial entre $t$ e $s_i$, $t_f$ conterá o
    \item $t_f$ conterá o restante de $t$ que não deu match com $s_i$, sendo que
    toda a sub árvore com a raiz em $t$, agora terá $t_f$ como raiz.
    \item um novo nó $h$ será criado a partir do resto de $s_i$ que não combinou
    com a substring em $t$. O nó $h$ será um nó folha, filho de $t_p$.
\end{itemize}

Por fim, pode-se afirmar que a montagem da árvore de sufixos tem um comportamento
assintótico em $O(n^2)$.
Essa afirmação é trivial, uma vez que é sabido que o custo da busca de uma
substring de tamanho $m$ em uma árvore de sufixos é $O(m)$.
Para inserir os $n$ sufixos do texto $Q$ em uma árvore de sufixos, deve-se
realizar $n$ buscas de tamanho $1, 2, 3, .... n-2, n-1, n$ -- uma para cada
sufixo, e então encontrar a posição que o novo nó será criado.
Assim, o custo dessa inserção é a soma de uma progressão aritmética e tem como
fórmula: 

$$\sum_{i=1}^{n}i = \frac{n(n-1)}{2} = O(\frac{n^2-n}{2}) = O(n^2)$$


\subsection{Encontrar a maior substring que se repete}
Nesse algoritmo, há a necessidade de uma execução recursiva, em que uma chamada
mais distante do topo da pilha de recursão deverá aplicar uma função
\textit{max} para as chamadas internas.
Para isso, o algoritmo escolhido foi uma busca em profundidade, em que o nó
pai da árvore deverá obter o \textit{max} dentre os nós filhos, mas com algumas
exceções.
A seguinte expressão recursiva demonstrará a execução da computação do DFS:

\[ biggest\_substr(t) = 
    \begin{cases}
        (repetitions = 0, size = 0) \quad \text{if t is a leaf node} \\
        \\
        (repetitions = \text{quantity(t.childrens)}, size = \text{t.size}) \\
        \quad \quad \quad \text{if all t.childrens are leaf nodes}\\
        \\
        (repetitions, size) = max(\text{for each biggest\_subtr(t.childrens)}) \\
         \quad \quad \quad \text{if one of t.childrens is not a leaf}
    \end{cases}
\]

Vale expressar que a função $max$, para cada retorno de $biggest\_substr()$,
realiza uma comparação entre a melhor substring atual e a obtida com o
retorno da função, então se o retorno for uma melhor escolha que o atual,
uma troca acontece.
Essa validação ocorre de forma a obter o melhor conjunto
$(repetitions, size)$ o qual é o par que tem maior valor $size$ e que
$repetitions > 1$, ou seja, a maior substring que acontece mais de uma vez.
Observe que a partir do momento em que $biggest\_substr()$ é chamada passando
a raiz da árvore como parâmetro, se a raiz não possui nós folha e nem é uma folha
por si só, então o terceiro caso será executado, chamando $biggest\_substr()$
para todos os filhos da raiz.
Esse fato acontece sucessivamente até que se chegue ao primeiro caso base (em
que o nó é uma folha) e se retorne para o segundo (no qual todos os filhos do
nó são folhas).

Pode-se dizer então que a função $max$ é executada quando o "desempilhamento"
das chamadas recursivas acontece, ou seja, é necessário que a busca em
profundidade que parte de $t$ retorne até $t$ para que o estado $t.(repetitions,
size)$ seja computado.
Assim, quando $t$ terminar as chamadas de $biggest\_substring()$ para todos os
filhos, obterá a melhor escolha.
De posse da melhor escolha $c = (rep, siz)$, a chamada que solicitou
$biggest\_substring()$ para $t$ deverá retornar o par $(c.rep, c.siz + t.size)$,
pois o tamanho de $t$ deve ser incrementado à melhor escolha.
A quantidade de repetições não mudará na computação de $biggest\_substring(t)$,
pois as repetições são computadas em função da quantidade de nós folha que a
sub árvore com raiz em $t$ possui.

A função de encontrar a maior string que se repete no texto é a codificação
das afirmações feitas acima.
Entretanto, para possibilitar a obtenção dos índices da string desejada, o
retorno da função não se trata de $(repetitions, size)$ mas sim $(repetitions,
indexes)$.
Nesse caso, $indexes$ se refere aos índices de inicio e fim da maior substring
que se repete e a partir desses índices é possível obter o tamanho apenas com
uma subtração.

Para determinar o comportamento assintótico da busca da maior substring que se
repete, serão tomadas duas afirmações como base.
Levando em consideração que a árvore fora escrita a partir de um texto $T$ de
tamanho $n$, ou seja, $T$ possui $n$ sufixos diferentes.
Dito isto, pode-se afirmar que:
\begin{enumerate}
    \item A quantidade de nós folha da árvore de sufixo é igual $n$.
    É possível verificar tal afirmação, validando que a construção da árvore
    consiste na criação de nós folha para cada sufixo de $T$.
    Uma vez que a busca da posição correta para adicionar um novo sufixo $s_i$
    parte do princípio de consumir o maior prefixo possível de $s_i$,
    o que restará após tal busca é um sufixo restante de $s_i$ (que também é um
    sufixo de $T$) e que se tornará uma nova folha.
    \item A quantidade de nós intermediários na árvore é no máximo igual a $n-1$.
    Pela criação da árvore, nunca haverá um nó intermediário que possua apenas um
    nó filho, pois a criação de nós intermediários parte do caso de borda citado
    na subseção anterior.
    Ou seja, um nó intermediário é criado apenas em \textit{splits}.
    Assim, para um nível $j$ da árvore com $m$ nós a quantidade máxima de nós
    no nível $j-1$ é igual a $m \div 2$, sendo o caso em que cada nó de $j-1$
    tem apenas dois filhos.
    Por fim, visto o caso da máxima existência de nós pai (2 filhos por nó),
    para a árvore completa de $n$ folhas com $l$ níveis, pode-se obter a
    quantidade de nós intermediários pela soma: $\sum_{i=0}^{l-1}2^i = 2^l-1 = n-1$,
    pois o $i$-ésimo nível de uma árvore binária completa, possui $2^{i-1}$ nós.
\end{enumerate}

Tomando as duas afirmações como base, e o algoritmo de busca em profundidade,
pode-se afirmar que $n$ nós folha serão visitados e no máximo $n-1$ nós
intermediários.
Isso significa que, uma vez que a visita a cada nó tem custo $O(1)$ o custo total
da busca de maior substring é $O(2n-1) = O(n)$.

\subsection{Custo de espaço}

As duas afirmações da subseção anterior, permitem admitir que no máximo $2n-1$
nó são criados.
Então, por hora, ignorando as arestas do grafo, o custo de memória $O(n)$.
Porém, como uma árvore é um grafo direcional acíclico, em que um nó não possui
mais de um nó "pai", a quantidade de arestas é igual a quantidade de nós na árvore
decrescido de $1$, pois o nó raiz não possui predecessor.
Logo a quantidade de arestas é no máximo $2n-2$.
Portanto o custo de memória do grafo é uma soma de $2n-1$ nós mais $2n-2$ arestas,
o que resulta em $O(n)$ para o custo de espaço.
