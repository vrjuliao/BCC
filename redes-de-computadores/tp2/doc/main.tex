 \documentclass[10pt]{extarticle}

%SOME USEFUL PACKAGES
\usepackage[portuguese]{babel}
\usepackage{extsizes}
\usepackage[reqno]{amsmath} % equation numbers on the right
\usepackage{amsmath}
\usepackage{amssymb}
\usepackage{amsthm}
\usepackage{amscd}
\usepackage{indentfirst}
\usepackage{amsfonts}
\usepackage{listings}
\usepackage{graphicx}
\usepackage{dsfont}
\usepackage{fancyhdr}
\usepackage{enumerate}
\usepackage[utf8]{inputenc}
\usepackage{booktabs}
\usepackage{listings}
\usepackage[margin=1in]{geometry}
\usepackage[hypcap]{caption}
\usepackage{longtable}
\usepackage{etoolbox}
%\usepackage[alf]{abntex2cite}
\usepackage[siunitx]{circuitikz}
\usepackage{tikz}
\usepackage{enumitem}
\usepackage{algorithm}
\usepackage{algorithmic}
\usepackage{hyperref}


\usepackage{verbatim}
\usepackage{float}

%HEADER
\pagestyle{fancy}\lhead{V. Julião Ramos} \rhead{DCC023 - Redes de Computadores - 2020.1}
\chead{{\large{\bf }}} \lfoot{} \rfoot{\bf \thepage} 
\cfoot{}


\title{ \textsc{Universidade Federal de Minas Gerais} \\
\textsc{Departamento de Ciência da Computação}\\
\bigskip TP2 - Servidor DNS}

\author{Vinicius Julião Ramos \\ 
\normalsize{2018054630} \\
\normalsize{viniciusjuliao@dcc.ufmg.br}}

\date{\today}

\begin{document}
\maketitle

\section{Introdução}

O objetivo desse trabalho é implementar um servidor DNS que tenha uma interface
de linha de comando.
Esse serviço é implementado com o protocolo UDP e para fornecer um servidor que
esteja em execução ao mesmo tempo que o leitor de linha de comando, dividiu-se
execução em dois subprocessos.
Tais subprocessos foram criados através da implementação de \textit{threads},
sendo que um deles realiza a "escuta" da rede, enquanto o outro recebe dados
pela linha de comando.
A parte que recebe requisições de rede realiza verifica se o endereço está presente
em seus dados, e caso não esteja, requisita o endereço para os outros servidores
DNS que são adicionados como forma de \textit{links}.

\section{Decisões de implementação}
Na solução apresentada por esse trabalho, um mesmo programa se encarrega de
executar cliente e servidor UDP.
A comunicação é feita de forma que a parte servidor é requisitado para responder
se conhece algum endereço IP de acordo com o \textit{hostname}.
No caso em que o programa não conheça o nome do \textit{host}, uma parte
cliente é chamada para requisitar a outros servidores DNS sobre tal
\textit{hostname}.

\subsection{Protocolo de comunicação}

Uma vez que protocolo de comunicação, foi descrito pela especificação do
trabalho, esse relatório descreverá as decisões de implementação.
Entretanto há três pontos que merecem atenção quanto ao padrão de comunicação:
\begin{enumerate}
    \item As mensagens trocadas entre servidor e cliente possuem um cabeçalho
          de 1 Byte, devendo conter o número inteiro \texttt{1} ou \texttt{2}.
          Esse valor não se trata de um \textit{string} contendo o caractere
          \texttt{'1'}, uma vez que esse valor na tabela ASCII seria convertido
          para o inteiro \texttt{49}. O mesmo acontece com o código \texttt{2}.
    \item Quando o servidor não encontra um endereço para determinado host,
          inclusive analisando os \textit{links}, deve-se retornar o
          \textit{payload} contendo \texttt{-1}. Nesse caso o corpo da mensagem
          possui uma \textit{string} \texttt{"-1"} e não um valor inteiro.
          Essa decisão foi tomada com base numa postagem no fórum de
          dúvidas da disciplina no Moodle.
    \item O endereço IP que o servidor envia para o cliente é uma \textit{string}
          em formato decimal. Ou seja, dado um endereço \texttt{192.0.0.1}, o
          servidor envia uma mensagem com a \textit{string} \texttt{"192.0.0.1"}.
\end{enumerate}


\subsection{Estruturas de dados}
Uma vez que a solução foi elaborada em C++, as estruturas utilizadas foram
providas pela biblioteca padrão da linguagem. Os \textit{hostnames} e seus
respectivos endereços são salvos em \texttt{std::map}\
\footnote{\href{http://www.cplusplus.com/reference/map/map/}\
{http://www.cplusplus.com/reference/map/map/}} em que as chaves são
os nomes e o valor é o endereço IP correspondente.
Esse detalhe de implementação faz com que a busca pelo endereço de um
\textit{host} tenha um custo menor em contraste ao uso de um \textit{array}.

Também era necessário armazenar os links que futuramente seriam utilizados para
encontrar os \textit{hostnames} que não estivessem presentes no mapa.
Então, criou-se uma \textit{struct} que armazena duas strings: endereço IP e
porta UDP em que outro servidor DNS estiver em execução.
Como podem existir múltiplos links, novamente um recurso da biblioteca padrão
de C++ foi utilizado, provendo um \texttt{std::vector}\
\footnote{\href{http://www.cplusplus.com/reference/vector/vector/}\
{http://www.cplusplus.com/reference/vector/vector/}}
que permite o armazenamento em forma de um \textit{array} dinâmicamente
gerenciado.

Tais estruturas instanciadas em forma de ponteiros, para que os dois
subprocessos possam acessá-las.
Portanto, há um trecho de memória comum a mais de uma função, sendo que apenas
a parte que recebe comandos pelo terminal podem alterar as informações dessas
estruturas.
Já o processo servidor utiliza o mapa e o vetor apenas como consulta que é
igualmente executada pelo comando \textit{search} da linha de comando.
Na thread que executa um socket ativo, existe uma instância da classe
\textbf{UDPServer}, a qual é responsável pelo recebimento das requisições de
\textit{search}. Essa classe implementa as funções de socket passivo POSIX.

\subsection{Comando \textit{search}}
Esse comando merece atenção pelo uso da classe \textbf{UDPClient}.
Neste ponto, a primeira etapa é validar junto ao mapa de \textit{hosts}, se
o nome pesquisado está presente ou não.
Uma vez que o host não é encontrado, então a segunda etapa se trata da criação
de múltiplas instâncias de \textbf{UDPClient} para requisitar a cada servidor
DNS (link) se estes possuem o \textit{hostname} pesquisado.
Tanto o subprocesso servidor quanto o subprocesso de comandos da linha de
comandos podem executar um \textit{search}.


\section{Execução}
Visto que o código foi escrito em C++, a compilação é feita por meio de um
arquivo \textit{Makefile}.
Então, usando uma máquina Linux, execute:
\lstset{language=bash}
\begin{lstlisting}[frame=single]
foo@bar:~/tp2$ make
\end{lstlisting}

Após a compilação, um arquivo executável será gerado (\textbf{servidor\_dns}).
O comando de execução é:
\lstset{language=bash}
\begin{lstlisting}[frame=single]
$ ./servidor_dns <porta udp> [arquivo-inicial]
\end{lstlisting}

O argumento obrigatório \texttt{<porta udp>} é o número da porta que o servidor
dns alocará.
Já o \texttt{[arquivo-inicial]} é um argumento opcional no qual, dado o arquivo
nomeado por esse parâmetro, o programa executará os comando presentes nele.
Após a execução dos comandos do arquivo, novos comandos poderão ser digitados
no terminal.

Um aspecto que merece atenção é que a versão do IP é parametrizado por um
atributo da classe \textbf{Utils}.
Caso seja necessário a mudança da versão do protocolo pode ser alterada nas
linhas iniciais do arquivo de código \texttt{main.cpp}

\section{Conclusão}
O trabalho prático foi útil para visualizar e entender a comunicação UDP e
também entender porquê o serviço de DNS utiliza tal protocolo.
A utilização de \textit{threads} e de estruturas de dados que compõem uma
memória compartilhada entre processos, mostrou a relação próxima entre
aplicações de rede e múltiplos subprocessos.

Ao tentar executar esse trabalho juntamente com o de outro colega de classe,
a implementação correta do protocolo se mostrou eficaz.
Uma vez que os dois trabalhos estavam de acordo com a especificação, os
programas conseguiram se comunicar perfeitamente. 
\end{document}