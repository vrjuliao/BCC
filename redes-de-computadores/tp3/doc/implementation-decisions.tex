\section{Decisões de Implementação}
\begin{figure}[h]
	\begin{center}
		\includegraphics[scale=0.55]{fig/uml.png}
	\end{center}
	\caption{\label{fig:digrama} Diagrama da estrutura de classes.}
\end{figure}

Essa seção trata de como as quatro principais funções, requeridas pela descrição
do trabalho, foram desenvolvidas.
Também é colocado em cheque as dificuldades referentes às soluções propostas.
Mas antes de descrever a implementação, é necessário abordar três estruturas de
dados presentes na classe RouterList:
\begin{itemize}
    \item \textbf{Route}: Atributo \textbf{ipaddr}: é o endereço final da rota.
    Atributo \textbf{last\_update}: é um atributo que salva a ultima vez em que
    a rota foi atualizada.
    Atributo \textbf{hops}: é um dicionário \textbf{(K\_addr, V)} em que
    \textbf{K\_addr} é o endereço ip do link associado ao roteador e \textbf{V}
    é o custo de enviar uma mensagem para o endereço \textbf{ipaddr} passando pelo
    link \textbf{K\_addr}.
    \item \textbf{RouterList.\_\_routes}: Trata-se de um dicionário
    \textbf{(R\_addr, R)}. Nesse caso \textbf{R\_addr} é o endereço de uma
    máquina e \textbf{R} trata-se da instância de \textbf{Route} correspondente
    a tal destino.
    \item \textbf{RouterList.\_\_links}: Dicionário \textbf{(L\_addr, L)} em
    que os links de endereço \textbf{L\_addr} são informados na entrada padrão,
    através do comando \texttt{add}, juntamente com o seu peso.
    Essa informação é armazenada em uma instãncia da classe \textbf{Link}.
\end{itemize}


\subsection{Atualizações Periódicas}
A cada meio segundo, é feito uma chamada ao método \texttt{Router.\_\_share\_router\_list}.
Então esse método trata de analisar se já passou tempo suficiente desde a
última atualização, para que uma nova atualização seja feita.
Caso seja necessário atualizar novamente, então as listas de distâncias 
são requisitados da tabela de roteamento.
Como essa lista de distâncias pode variar de acordo com o link, devido ao
\textit{Split Horizon}, o retorno dessa chamada é um dicionário, em que cada
elemento \textbf{(addr, dist)} corresponde à chave \textbf{addr} (endereço) do
link e \textbf{dist} é a respectiva lista de distancias.

O método itera sobre o dicionário e envia uma mensagem de \texttt{update}
contendo a respectiva lista de distâncias de cada um dos vizinhos.
Observe que o método recebe um atributo \textbf{past}, que é um
\texttt{datetime} da ultima vez em que uma atualização foi enviada.
Caso uma atualização seja feita, então retorno \texttt{True} informa que o
atributo \textbf{past} deve ser atualizado para o momento corrente.

\lstinputlisting[language=Python,
label=code:share-router-list,
caption=Envio das atualizações periódicas das listas de roteamento]
{codes/share_routes.py}

\subsection{Remoção de rotas desatualizadas}
Essa funcionalidade é desempenhada pela classe \textbf{RouterList}.
Como tal classe é a única que possui o controle para inserção, remoção e
atualização das rotas, é natural que esses trabalho seja executado pela tabela
de roteamento.
Ao instanciar um objeto \textbf{RouterList}, deve-se informar qual o tempo de
vida de uma rota desatualizada.
Isso serve para que cada rota criada, tenha um \textit{timer} associado, e esse
contador de tempo consiga analisar se o tempo de vida foi extrapolado ou não.
Observe que objetos \textbf{Route} implementa seu próprio \textit{timer} pelo
atributo \textbf{last\_update}.

O período de remoção das rotas desatualizadas é igual a $4 \times period$, em
que $period$ é o tempo de atualização das rotas dos links vizinhos.
Como no método \texttt{\_\_share\_router\_list} há um controle sobre $period$,
optou-se por implementar uma chamada ao método \texttt{RouterList.refresh\_routes}
no mesmo controlador do envio das listas de distâncias; assim como mostra a
Linha 7 do Listing \ref{code:share-router-list}.
Sendo assim, a cada $period$ segundos, valisa-se se há rotas desatualizadas.
Isso se dá pelo seguinte código:
\lstinputlisting[language=Python,
label=code:refresh-routes,
caption=Remoção das rotas desatualizadas]
{codes/refresh_routes.py}

Há outro fator que merece atenção: Quando uma mensagem do tipo \texttt{update}
chega até o roteador, então a tabela de rotas deve ser atualizada.
Nessa atualização da tabela de rotas, todas aquelas rotas contidas na mensagem
têm seu \textit{timer} atualizado para o momento corrente.
Isso faz com que essa rota ganhe mais tempo de vida.

\subsection{Rerroteamento imediato}
É importante ressaltar que todo roteamento executado pelo servidor é feito de
forma imediata.
Ou seja, a melhor rota para determinado destino é calculada apenas quando uma
mensagem for enviada para tal endereço.
A troca de mensagens é desempenhada pela classe \textbf{Router}, sendo que
qualquer envio de dados é executado pelo método \texttt{Router.\_\_send\_message\_as\_json}. 
Observando o Listing \ref{code:route}, a cadeia de chamadas que retorna a melhor
rota inicia-se na Linha 16, em que o endereço de destino é informado para
\texttt{Router.next\_hop}.
Nesse ultimo método, caso não haja uma rota para tal endereço, então um valor
\texttt{None} é retornado, mas caso contrário, requisita-se o cálculo da rota
para classe \textbf{Route}.
Por fim, \texttt{Route.route} retorna o endereço da rota que possui menor peso.

\lstinputlisting[language=Python,
label=code:route,
caption=Retorna a melhor rota]
{codes/route.py}

Na abordagem aplicada para esse calculo de rerroteamento mesmo quando uma rota
é removida ou alterada da tabela de roteamento, as demais rotas permanecem
intactas.
Logo, para o caso em que a melhor rota para um endereço $X$ fora removida,
quando \textbf{Router} desejar enviar outra mensagem para $X$, uma nova melhor
rota será obtida pela chamada de \texttt{RouterList.next\_hop(X)}

\subsection{Spilt Horizon}
O método \textit{Split Horizon} é aplicado na geração dos dicionários
\textit{distances} que posteriormente serão adicionados às mensagens.
Antes de gerar o dicionário \textbf{distances} que associa o endereço do
vizinho à sua lista de distâncias, gera-se um dicionário contendo todas as
melhores rotas (\textbf{all\_distances}).
Então, para gerar \textbf{distances[link]} -- que é o dicionário de distâncias
de determinado link -- itera-se sobre todos os vizinhos, de forma que
\textbf{distances[link]} recebe todas as rotas que obedecem à optimização
\textit{Split Horizon}.
Ou seja, \textbf{distences[link]} recebe um dicionário contendo as rotas as
quais o endereço de \textbf{link} não seja dado como o melhor caminho.

\lstinputlisting[language=Python,
label=code:all-distances,
caption=Retorna o dicionário de distâncias usando Split Horizon]
{codes/split_horizon.py}
