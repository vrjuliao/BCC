\section{Introdução}

Para possibilitar a virtualização de um roteador, através de um aplicação,
necessitou-se utilizar um protocolo da camada de transporte.
Nesse cenário, optou-se pelo UDP, uma vez que esse protocolo é o que mais se
aproxima da camada de rede, pelo motivo de não fornecer quaisquer garantias
de entrega, controle de erros ou gerenciamento de conexão.
Portanto, para esse socket, uma classe \textbf{Router} foi criada com a
finalidade de gerenciar a troca de informações.
Essa mesma classe é responsável por executar comandos em duas \textit{thread}
diferentes, sendo que uma delas recebe comandos da entrada padrão do sistema e 
a outra executa o protocolo de rede especificado pelo trabalho.

A opção de criar uma classe separada para gerenciar a tabela de roteamento
permitiu gerênciar todas a funções que cabem à uma tabela de roteamento: a
remoção de rotas desatualizadas, a montagem da lista de distâncias através
do \textit{Split Horizon} e também o rerroteamente imediato. Essa última função
é desempenhada por demanda, o que permite que a melhor rota para determinado
endereço seja calculada apenas quando o roteador envia mensagem para tal
destino. Já a atualização periódica de rotas é uma atividade conjunta entre o
roteador e a tabela de roteamento; a tabela de roteamento entrega ao roteador
a lista de distâncias, mas cabe ao roteador atualizar os vizinhos de acordo com
o período especificado.